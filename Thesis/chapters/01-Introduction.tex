\chapter{01-Introduction}
\label{ch:introduction}


The task of linking names in free text to referent entities in a knowledge base is known as entity linking. Entity linking can assist text analysis algorithms in comprehending the context of the name in depth by using known entity information. The most recently proposed linking systems are divided into two steps: candidate creation (aka. candidate generation) and candidate rating (aka. entity disambiguation).\newline
Numerous works have been done to improve the systems that perform the task of entity linking, and many of these works have focused on the second step of the named task[cite 01]. The 
developped systems have achieved considerable results, e.g. in [cite 02] they achieved an accuaracy of 95\% of five standard entity disambiguation datasets.\newline
Nonetheless, regardless of how accurate the system is, the first stage of candidate creation is critical to a solid performance, since the abscence of the actual referred entity in the candidate set generated will inevitably lead to erronous output.\newline
In this paper, we discuss the strategy we use to generate candidates for mentions in a given text. This strategy is similar to the one described in [cite 03+04]'s publications and others. This technique leverages knowledge graph embeddings as a target space to which the embeddings of the mentions will be mapped to. \newline
bla \newline
bla \newline
blop \newline

The rest of the thesis is organized as follows: ....
